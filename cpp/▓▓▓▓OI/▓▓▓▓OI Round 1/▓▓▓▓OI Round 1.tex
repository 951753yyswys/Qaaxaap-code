\documentclass[UTF8,a4paper]{ctexart}
\usepackage{geometry}
\usepackage{multicol}
\usepackage{multirow}
\usepackage{tabu}
\usepackage{xeCJK}
\usepackage{CJK}     
\usepackage{xeCJKfntef}                     
\usepackage{fancyhdr}               
\usepackage{graphicx}                 
\usepackage{lastpage}    
\usepackage{listings}
\usepackage{xcolor}
\usepackage{fontspec}
\usepackage{layout}
\usepackage{titletoc}
\usepackage{mathrsfs}
\usepackage{framed}
\usepackage[colorlinks,linkcolor=blue]{hyperref} 
\newcommand\filename[1]{\emph{\textbf{#1}}}
\newfontfamily{\consolas}{Consolas}
\newcommand\udot[1]{\textbf{\color{black}\CJKunderdot{\color{black}#1}}} % 第一个 color 调整加粗字体下着重号的颜色
\newcommand\newprob[2]{
	\newpage
	\pagestyle{fancy}
	\lhead{{\consolas ▒▒▒▒}OI Round2 Div2.5} \rhead{#1(#2)}
	\cfoot{第 \thepage 页 \qquad 共 \pageref{LastPage} 页}
	\phantomsection
	\addcontentsline{toc}{section}{#1(#2)}
	\begin{center}
		\LARGE
		\textbf{#1(}#2\textbf{)}
	\end{center}
	\large
	%
	\textbf{【题目背景】}
	\phantomsection
	\addcontentsline{toc}{subsection}{【题目背景】}
}
\newcommand\para[1]{
	$ $ \\ 
	\textbf{【#1】}
	\phantomsection
	\addcontentsline{toc}{subsection}{【#1】}
}
\newcommand\sample[2]{
	$ $ \\ 
	\textbf{【样例} #1\textbf{#2】}
	\phantomsection
	\addcontentsline{toc}{subsection}{【样例 #1 #2】}
}
\lstset{
	basicstyle={      
		\color{black}
		\fontspec{Consolas}
	},
	keywordstyle={
		\color{blue}
		\fontspec{Consolas}
	},
	numberstyle={
		\color{gray}
		\texttt
	},
	rulecolor=\color{blue},
	numbers=left,                               
	frame=single,                            
	% frameround=tttt,
	morekeywords={Sample, Input, Output},   % 可以手动添加关键字
}
\setmonofont{Consolas}
\geometry{left=2.52cm,right=2.52cm,top=2.5cm,bottom=2.5cm}
\begin{document}
	\pagestyle{fancy}
	\lhead{{\consolas ▒▒▒▒}OI Round2 Div2.5} \rhead{}
	\cfoot{第 \thepage 页 \qquad 共 \pageref{LastPage} \color{black} 页}
	\thispagestyle{empty}
	\addcontentsline{toc}{section}{注意事项}
	\begin{center}
		\Huge
		\textbf{{\consolas ▒▒▒▒}OI Round2 Div2.5}
		\\
		\Huge 
		{\consolas ▒▒▒▒▒} OI Contest
		\\
		\huge
		\textit{\consolas ▒▒▒▒▒▒▒}
		\\
		\Large
		\textbf{时间:}{\consolas ▒▒▒▒}\textbf{年}{\consolas ▒▒}\textbf{月}{\consolas ▒▒}\textbf{日} {\consolas ▒▒:▒▒} $\sim$ {\consolas ▒▒:▒▒}
		\\
	\end{center}
	\large
	\begin{center}
		\begin{tabular}{|p{3.1cm}|p{2.5cm}|p{2.5cm}|p{2.5cm}|p{2.5cm}|}
			\hline
			题目名称 & 小木棍 & 献祭 & 染色体 & 线段树二分 \\
			\hline
			题目类型 & 传统型 & 传统型 & 传统型 & 传统型 \\
			\hline
			目录 & \texttt{polygon} & \texttt{xor} & \texttt{lx} & \texttt{segbs} \\
			\hline
			可执行文件名 & \texttt{polygon} & \texttt{xor} & \texttt{lx} & \texttt{segbs} \\
			\hline
			输入文件名 & \texttt{polygon.in} & \texttt{xor.in} & \texttt{lx.in} & \texttt{segbs.in} \\
			\hline
			输出文件名 & \texttt{polygon.out} & \texttt{xor.out} & \texttt{lx.out} & \texttt{segbs.out} \\
			\hline
			每个测试点时限 & 1.0 秒 & 1.0 秒 & 1.0 秒 & 2.0 秒 \\
			\hline
			内存限制 & 256 MiB & 256 MiB & 256 MiB & 256 MiB \\
			\hline
			预估{\consolas▒▒▒▒} & {\consolas ▒▒▒▒} & {\consolas ▒▒▒▒} & {\consolas ▒▒▒▒} & {\consolas ▒▒▒▒} \\
			\hline
			测试点数目 & 20 & 5 & 5 & 60 \\
			\hline
			测试点是否等分 & 是 & 是 & 是 & 是 \\
			\hline
		\end{tabular}
	\end{center}
	提交源程序文件名
	\begin{center}
		\begin{tabular}{|p{3.1cm}|p{2.5cm}|p{2.5cm}|p{2.5cm}|p{2.5cm}|}
			\hline
			对于 C++ \  语言 & \texttt{ltm.cpp} & \texttt{xor.cpp} & \texttt{lx.cpp} & \texttt{segbs.cpp} \\
			\hline
		\end{tabular}
	\end{center}
	编译选项
	\begin{center}
		\begin{tabular}{|p{3.1cm}|p{11.2cm}<\centering|}
			\hline
			对于 C++ \ 语言 & \texttt{-O2 -std=c++14 -static -m32} \\
			\hline
		\end{tabular}
	\end{center}
	\udot{注意事项(请仔细阅读)} 
	\\
	\indent
	1. 本场比赛为阶段学习评估,总体难度较小,与 {\consolas ▒▒▒▒▒▒▒▒▒}-Div2.5 难度相近。\par
	2. 由于评测机性能原因,评测时题目时间限制可能会进行调整。\par
	3. 时间限制、空间限制分别不小于标准程序运行时间和内存使用的 1.1 倍、1.1 倍。\par
	4. 本场比赛可支持 C++14 语言。\par
	5. \udot{本场比赛不提供{\consolas ▒▒▒▒▒▒▒▒}。} \par
	6. 本校模拟赛题不得外传,包括但不限于:私自拷题,在任何网站记录题目信息(包括洛谷非公开云剪贴板,非公开题目等),向在线 AI 询问题目,或{\consolas ▒▒▒▒▒▒▒▒▒▒▒▒▒▒}。\par
	7. 题目背景为架空世界,请勿将现实中人物对号入座,否则 {\consolas ▒▒▒▒▒▒▒▒▒▒▒▒▒▒▒▒▒}。\par
	8. 300pts+ 并且猜出题目中的歌曲者可向出题人领赏 $30$RMB。 \par
	9. 请把 {\consolas ▒▒▒▒} 读作【数据删除】。\par 
	10. {\consolas ▒▒▒▒}造数据,欢迎各种乱搞,各位选手 AK 后请勿大声喧哗。
	%%%%%%%%%%%%%%%%%%%%%%%%%%%%%%%%%%%%%%%%%%%%%%%%%%%%%%%%%%%%%%%%
	\newprob{小木棍}{polygon} \\ \indent
	九点一刻。\par
	{\consolas ▒▒▒▒▒▒▒} 盯着刚刚通过的 polygon 大样例,十分的无聊。\par
	为了让自己能安心睡着,{\consolas ▒▒▒▒▒▒▒} 给自己出了一道 polygon-Pro-Max-Ultra 版。\par 
	由于他真的盯着 polygon-Pro-Max-Ultra 睡着了,所以问题就留给你了。
	\para{题目描述} \\ \indent
	给你 $n$ 根小木棍,长度 $a_1,a_2,\dots,a_n$。
	你需要选出一个可重集 $s$,规则如下:\par 
		1. 将 $a$ 分成一些可重集 $x_1,x_2,\dots,x_k$ 使 $a$ 中任意元素都恰好属于某一个可重集。\par
		2. 选择 $x_i$ 的众数加入 $s$ $(i\in [1,k])$。\par
	现在问你不同的 $s$ 有多少种,答案对 $(114514+1919810)$ 取模。
	\para{输入格式} \\ \indent
	从文件 \filename{polygon.in} 中读入数据。 \par
	第一行一个整数 $n$,表示木棍个数。\par 
	第二行 $n$ 个整数,第 $i$ 个整数表示 $a_i$。
	\para{输出格式} \\ \indent
	输出到文件 \filename{polygon.out} 中。 \par
	一行一个正整数,表示答案。
	\sample{1}{输入}
	\begin{lstlisting}
3
1 2 2
	\end{lstlisting}
	
	\sample{1}{输出}
	\begin{lstlisting}
4
	\end{lstlisting}
	
	\sample{1}{解释}\par 
	所有可能的 $s$ 有如下四种情况:$\{2\},\{2,2\},\{1,2\},\{1,2,2\}$ \par 
	\sample{2}{输入}
	\begin{lstlisting}
10
1 1 1 2 2 2 3 3 3 4
	\end{lstlisting}
	
	\sample{2}{输出}
	\begin{lstlisting}
126
	\end{lstlisting}
	
	\sample{3}{} \\ \indent
	见选手目录下的 \filename{polygon/polygon3.in} 与 \filename{polygon/polygon3.ans},该样例满足 $n=5000$。\par
	
	\para{数据范围} \\ \indent
	\begin{center}
		\begin{tabu}{c|c|c}
			\tabucline[2pt]{-}
			测试点编号 & 特殊性质 & $n$ \\ \tabucline[1.2pt]{-}
			$1$ & 无 & \multirow{2}{*}{5}  \\ \cline{1-2}
			$2$ & \multirow{2}{*}{A} &  \\ \cline{1-1}\cline{3-3} 
			$3$ &  & \multirow{2}{*}{50} \\ \cline{1-2} 
			$4$ & \multirow{2}{*}{无} & \\ \cline{1-1}\cline{3-3} 
			$5$ &  & \multirow{2}{*}{500} \\ \cline{1-2} 
			$6$ & \multirow{2}{*}{A} &  \\ \cline{1-1}\cline{3-3} 
			$7$ &  & \multirow{2}{*}{5000} \\ \cline{1-2} 
			$8\sim 20$ & 无 & \\  \cline{1-3}
			\tabucline[2pt]{-}
		\end{tabu}
	\end{center}
	
	对于 $100\%$ 的测试点,保证 $n\leq 5000,a_i\leq n$ $(i\in[1,n])$
	%%%%%%%%%%%%%%%%%%%%%%%%%%%%%%%%%%%%%%%%%%%%%%%%%%%%%%%%%%%%%%%%
	\newprob{献祭}{xor} \\ \indent
	\udot{本题难度预估:下位绿}。\par
	$------------------------ $ \par
	我知道你肯定是做第一题看不下去了才跑过来看第二题的。 \par 
	你现在如果再去看一眼第三题,你可能会丧失更多对这套题的兴趣。  \par
	但是这道题真的是一道水题。  \par
	而且第四题比这道题还要水。  \par
	就算你依然不会,你也应该认真思考,如何找到题目的突破口。  \par
	这对于思维训练的帮助是巨大的。  \par
	这也是我们打模拟赛的根本原因。  \par
	
	
	\para{题目描述} \\ \indent
	异或之神的面前,出现了 $n$ 个献祭者。  \par
	每个献祭者体内都用一种神秘物质:异或之液。\par
	每个献祭者都有一个编号以及异或之液的纯度,编号从 $1$ 到 $n$ 连续。 \par 
	这些献祭者按照编号从小到大被排成一排。  \par
	共有 $m$ 次献祭仪式,每一次献祭仪式,异或之神都会选择若干名编号连续的献祭者,其中编号最小的为 $l$,编号最大的为 $r$,且 $r-l>1$。\par
	然后,他会将选出的这些献祭者按照编号从小到大抽取他们的异或之液,并按顺序注入一条可以玻璃管内。  \par
	注意,不同献祭者的异或之液\udot{目前}是不会相融的。  \par
	注入完之后会把玻璃管切成若干段,注意,他只会在两种纯度不同的异或之液的交界处进行切割。  \par
	切割后,玻璃管被分成 $k$ 段,并且 $k\ge2$。  \par
	紧接着,异或之神就会使用魔法将每一段内的异或之液开始融合。  \par
	融合后的异或之液的纯度为融合前所有异或之液纯度的异或和。  \par
	当每一段中的异或之液都充分融合后,异或之神开始检查每一段的纯度。  \par
	如果所有段的纯度都相等,那么异或之神就会得到异或之液的法力,从而提升自己的异或水平,否则他将不会有任何变化。  \par
	现在,他想知道,对于给定的 $l,r$ 存不存在一种切割方案使得他可以提升自己的异或水平。 \par 
	如果存在输出 poly,否则输出 modui。  \par
	\para{输入格式} \\ \indent
	从文件 \filename{xor.in} 中读入数据。 \par
	第一行包含两个整数 $n$ 和 $m$。\par
	
	下一行包含 $n$ 个整数 $a_1,\cdots,a_n$。\par
	
	接下来的每行 $q$ 包含两个整数 $l$ 和 $r$,分别描述查询。\par
	\para{输出格式} \\ \indent
	输出到文件 \filename{xor.out} 中。 \par
	对于每个查询,输出 "poly" 或 "modui"。
	
	\sample{1}{输入}
	\begin{lstlisting}
		4
		5 5
		1 1 2 3 0
		1 5
		2 4
		3 5
		1 3
		3 4
		5 5
		1 2 3 4 5
		1 5
		2 4
		3 5
		1 3
		2 3
		7 4
		12 9 10 9 10 11 9
		1 5
		1 7
		2 6
		2 7
		11 4
		0 0 1 0 0 1 0 1 1 0 1
		1 2
		2 5
		6 9
		7 11
	\end{lstlisting}
	
	\sample{1}{输出}
	\begin{lstlisting}
		YES
		YES
		NO
		NO
		NO
		
		YES
		NO
		NO
		YES
		NO
		
		NO
		NO
		NO
		NO
		
		YES
		NO
		YES
		YES
		
	\end{lstlisting}
	\sample{1}{解释} \\ \indent
	本题没有多组数据,方便排版将 $4$ 组样例压在同一样例中。
	\para{数据范围} \\ \indent
	对于 $20\%$ 的数据,$n,m\le 100$。  \par 
	对于 $100\%$ 的数据,$n,m\le 2\times10^5$。 \par 
	%%%%%%%%%%%%%%%%%%%%%%%%%%%%%%%%%%%%%%%%%%%%%%%%%%%%%%%%%%%%%%%%%%%%%%%
	\newprob{染色体}{lx} \\ \indent
	\begin{center}
		蒹葭苍苍,白露为霜。\par
		所谓伊人,在水一方。\par
		溯洄从之,道阻且长。\par
		溯游从之,宛在水中央。\par
		蒹葭萋萋,白露未晞。\par
		所谓伊人,在水之湄。\par
		溯洄从之,道阻且跻。\par
		溯游从之,宛在水中坻。\par
		蒹葭采采,白露未已。\par
		所谓伊人,在水之涘。\par
		溯洄从之,道阻且右。\par
		溯游从之,宛在水中沚。\par
	\end{center}
	\para{题目描述} \\ \indent
	\udot{本题难度预估:下位紫}。\par
	$------------------------ $ \par
	\udot{本题有多组数据。}\par
	白露再次凝成霜,一年又一年,来自 ZJ 的 LX 大佬却不会长大,因为他有着像李渊一样的心智,不过嘛,终究还是要来到初中。\par 
	踏入新的省份,新的班级,LX 大佬是多么的陌生,望着周围新的环境、新的同学、新的老师以及那新的关系,他有些迷茫,但他那李渊一样的心智帮了他,让他渐渐地融入了这个新的班级。\par
	不过,LX 大佬引起了有个同学 Fish 的注意,因为 LX 是那么的活泼,和谁都能玩在一起,又是那么的优秀,常年霸榜\udot{逆序铜牌},Fish 不知怎的,似乎有种莫莫名名的感觉在她的内心荡漾。\par
	渐渐地,他们也熟识了,从陌生人到交往,从交往到熟识,从熟识到朋友,从朋友到……\par
	随着时间的推移,LX 和 Fish 也开始了信件上的单独相约,从生快的祝福,到绚丽的诗文,他享受着 Fish 信件的陪伴。  \par
	可是 Fish 不知道的是,LX 大佬从出生起,就比常人多了一条 $21$ 号染色体。 \par 
	基因的缺陷始终令 LX 自卑,不敢直面自己,看到作为正常入的 Fish 愉悦的活着的时候,他总感到难以自拔的悲伤。  \par
	明智的 LX 大佬意识到,他必须学会拒绝 Fish。  \par
	当 LX 大佬向 Fish 坦白一切时,Fish 彻底崩溃。  \par
	从此以后,Fish 的信件中遍充斥着血泪的控诉与呼喊,可 LX 仍然不为所动,再未回过一封。 \par  
	直到某一天,一封信引起了 LX 的注意。  \par
	这封信写得工工整整,恰好写满了 $n$ 行,$n$ 列的作文纸。  \par
	待他仔细看向信件的内容时,却看作文纸的每个格子中,都是一个数字。  \par
	学过 MO 的 LX 大佬一眼看出了 Fish 的把戏。\par  
	第 $i$ 行,第 $j$ 列的数中就是 $f(\gcd(i,j))$。\par  
	$f(i)$ 是斐波那契数列的第 $i$ 项。\par
	Fish 在信件的背面写道:\par
	\begin{framed}
		To LX:\par
		所有格子中数字的乘积,模 $998244353$,就是你所期待的,安史之乱的密码。  
		祝好。\par
	\end{framed}
	高傲的 LX 大佬却热泪盈眶。  \par
	他不敢相信,Fish 竟然为了他,不知付出多少努力,找到了消去他那可恶的染色体的关键数字。  
	可惜,计算那 $n^2$ 个数的乘积实在是过于庞大,他难以计算。\par
	所以他找到了你。  \par
	$ $\par
	形式化题意:\par
	给你一个 $n\times n$ 的方阵 $A$,$A_{i,j}=\gcd(f(i),f(j))$,请你求出 $A$ 中所有数的乘积模 $998244353$ 的值。 \par 
	
	\para{输入格式} \\ \indent
	从文件 \filename{lx.in} 中读入数据。 \par
	
	\para{输出格式} \\ \indent
	输出到文件 \filename{lx.out} 中。 \par
	Output. 
	
	\sample{1}{输入}
	\begin{lstlisting}
		Sample Input.
	\end{lstlisting}
	
	\sample{1}{输出}
	\begin{lstlisting}
		Sample Output.
	\end{lstlisting}
	
	\sample{1}{解释} \\ \indent
	Note.
	
	\sample{2}{} \\ \indent
	见选手目录下的 \filename{lx/lx2.in} 与 \filename{lx/lx2.ans}。
	
	\para{数据范围} \\ \indent
	对于 $60\%$ 的数据,$n,m\le 10^5$。 \par 
	对于 $100\%$ 的数据,$n,m\le 3\times10^7$。 \par 
	%%%%%%%%%%%%%%%%%%%%%%%%%%%%%%%%%%%%%%%%%%%%%%%%%%%%%%%%%%%%%%%%%%%%%%%%%%%
	\newprob{线段树二分}{segbs} \\ \indent
	\udot{题目名称与本题做法没有关联。}\par
	某场模拟赛正在进行。\par
	Uit:T1 怎么做啊? \par
	(T1 正解是莫队) \par
	Col:用树剖写。  \par
	十五分钟后。 \par 
	Uit:树剖过了。\par    
	
	\para{题目描述} \\ \indent
	\udot{本题难度预估:上位黄}。\par
	$------------------------$ \par
	给你一个长度为 $n$ 的序列 $a$,$m$ 次操作。\par 
	操作 $1$:给定 $l,r$,问 $a_l,a_{l+1},\cdots,a_r$ 是否两两不同。\par
	操作 $2$:给定 $l,r$,问 $a_l,a_{l+1},\cdots,a_r$ 是否两两相同。\par
	操作 $3$:给定 $l,r$,问 $a_l,a_{l+1},\cdots,a_r$ 是否是一个从 $1$ 到 $r-l+1$ 的排列。\par        
	操作 $4$:给定 $l,r,s,t$,问可重集 $\{a_l,a_{l+1},\cdots,a_r\}$,$\{a_s,a_{s+1},\cdots,a_t\}$ 是否完全相同。\par   
	顾名思义,可重集就是可以有相同元素的集合。
	
	
	\para{输入格式} \\ \indent
	从文件 \filename{segbs.in} 中读入数据。 \par
	
	\para{输出格式} \\ \indent
	输出到文件 \filename{segbs.out} 中。 \par
	Output. 
	
	\sample{1}{输入}
	\begin{lstlisting}
		Sample Input.
	\end{lstlisting}
	
	\sample{1}{输出}
	\begin{lstlisting}
		Sample Output.
	\end{lstlisting}
	
	\sample{1}{解释} \\ \indent
	Note.
	
	\sample{2}{} \\ \indent
	见选手目录下的 \filename{segbs/segbs2.in} 与 \filename{segbs/segbs2.ans}。
	
	\para{数据范围} \\ \indent
	对于 $25\%$ 的数据,$n,m\le 10^3$。  \par 
	对于 $50\%$ 的数据,$n,m\le 10^5$。 \par 
	对于 $75\%$ 的数据,$n,m\le 10^6$。 \par 
	对于 $100\%$ 的数据,$n,m\le 3\times10^7$。 \par   
	对于每一档不同的 $n,m$ 范围,都有 15 个点,分别对应 $4$ 个操作的所有组合。
\end{document}