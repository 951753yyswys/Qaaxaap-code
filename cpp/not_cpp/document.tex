\documentclass[UTF8,a4paper]{ctexart}
\usepackage{geometry}
\usepackage{multicol}
\usepackage{multirow}
\usepackage{tabu}
\usepackage{xeCJK}
\usepackage{CJK}     
\usepackage{xeCJKfntef}                     
\usepackage{fancyhdr}               
\usepackage{graphicx}                 
\usepackage{lastpage}    
\usepackage{listings}
\usepackage{xcolor}
\usepackage{fontspec}
\usepackage{layout}
\usepackage{titletoc}
\usepackage{mathrsfs}
\usepackage{framed}
\usepackage[colorlinks,linkcolor=blue]{hyperref} 
\newcommand\filename[1]{\emph{\textbf{#1}}}
\newcommand\udot[1]{\textbf{\color{black}\CJKunderdot{\color{black}#1}}} % 第一个 color 调整加粗字体下着重号的颜色
\newcommand\newprob[2]{
	\newpage
	\pagestyle{fancy}
	\lhead{CyreneOI Round2 Div2.5} \rhead{#1(#2)}
	\cfoot{第 \thepage 页 \qquad 共 \pageref{LastPage} 页}
	\phantomsection
	\addcontentsline{toc}{section}{#1(#2)}
	\begin{center}
		\LARGE
		\textbf{#1(}#2\textbf{)}
	\end{center}
	\large
	%
	\textbf{【题目背景】}
	\phantomsection
	\addcontentsline{toc}{subsection}{【题目背景】}
}
\newcommand\para[1]{
	$ $ \\ 
	\textbf{【#1】}
	\phantomsection
	\addcontentsline{toc}{subsection}{【#1】}
}
\newcommand\sample[2]{
	$ $ \\ 
	\textbf{【样例} #1\textbf{#2】}
	\phantomsection
	\addcontentsline{toc}{subsection}{【样例 #1 #2】}
}
\lstset{
	basicstyle={      
		\color{black}
		\fontspec{Consolas}
	},
	keywordstyle={
		\color{blue}
		\fontspec{Consolas}
	},
	numberstyle={
		\color{gray}
		\texttt
	},
	rulecolor=\color{blue},
	numbers=left,                               
	frame=single,                            
	% frameround=tttt,
	morekeywords={Sample, Input, Output},   % 可以手动添加关键字
}
\setmonofont{Consolas}
\geometry{left=2.52cm,right=2.52cm,top=2.5cm,bottom=2.5cm}
\begin{document}
	\pagestyle{fancy}
	\lhead{CyreneOI Round2 Div2.5} \rhead{}
	\cfoot{第 \thepage 页 \qquad 共 \pageref{LastPage} \color{black} 页}
	\thispagestyle{empty}
	\addcontentsline{toc}{section}{注意事项}
	\begin{center}
		\Huge
		\textbf{CyreneOI Round2 Div2.5}
		\\
		\Huge 
		MSANPU OI Contest
		\\
		\huge
		\textit{$\mathrm{Qaaxaap}$}
		\\
		\Large
		\textbf{时间:}2025\textbf{年}7\textbf{月}25\textbf{日} 14:00 $\sim$ 18:00
		\\
	\end{center}
	\small
	\begin{center}
		\begin{tabular}{|p{3cm}|p{1.7cm}|p{1.7cm}|p{1.7cm}|p{1.7cm}|p{1.7cm}|p{1.7cm}|}
			\hline
			题目名称 & Tribios &SkeMma720 & EpieiKeia216&Hyacinthia & NeiKos496 &PhiLia093\\
			\hline
			题目类型 & 传统型 & 传统型 & 传统型 & 传统型&传统型&传统型 \\
			\hline
			目录 & \texttt{Tribios} & \texttt{Anaxagoras} & \texttt{Castorice} & \texttt{Hyacinthia} &\texttt{Phainon}&\texttt{Cyrene}\\
			\hline
			可执行文件名 & \texttt{Tribios} & \texttt{Anaxagoras} & \texttt{Castorice} & \texttt{Hyacinthia} &\texttt{Phainon}&\texttt{Cyrene}\\
			\hline
			输入文件名(.in) & \texttt{Tribios} & \texttt{Anaxagoras} & \texttt{Castorice} & \texttt{Hyacinthia} &\texttt{Phainon}&\texttt{Cyrene}\\
			\hline
			输出文件名(.out) & \texttt{Tribios} & \texttt{Anaxagoras} & \texttt{Castorice} & \texttt{Hyacinthia} &\texttt{Phainon}&\texttt{Cyrene}\\
			\hline
			每个测试点时限 & 1.0 秒 & 1.0 秒 & 1.0 秒 & 2.0 秒 &1.0秒&1.0秒\\
			\hline
			内存限制 & 256 MiB & 256 MiB & 256 MiB & 256 MiB&256MiB&256MiB \\
			\hline
			测试点数目 & 10 & 10 & 20 & 20&10&10 \\
			\hline
			测试点是否等分 & 是 & 是 & 是 & 是 &是&是\\
			\hline
			是否使用子任务 &否&是&否&是&是&否\\
			\hline
		\end{tabular}
	\end{center}
	\large
	提交源程序文件名(.cpp)
	\small
	\begin{center}
		\begin{tabular}{|p{3cm}|p{1.7cm}|p{1.7cm}|p{1.7cm}|p{1.7cm}|p{1.7cm}|p{1.7cm}|}
			\hline
			对于 C++ 语言 & \texttt{Tribios} & \texttt{Anaxagoras} & \texttt{Castorice} & \texttt{Hyacinthia} &\texttt{Phainon}&\texttt{Cyrene}\\
			\hline
		\end{tabular}
	\end{center}
	\large
	编译选项
	\small
	\begin{center}
		\begin{tabular}{|p{3cm}|p{12.4cm}<\centering|}
			\hline
			对于 C++ 语言 & \texttt{-O2 -std=c++14 -static } \\
			\hline
		\end{tabular}
	\end{center}
	\large
	\udot{注意事项(请仔细阅读)} 
	\\
	\indent
	1. 本场比赛为阶段学习评估,总体难度较小,与 Codeforces-Div2.5 难度相近。\par
	2. 由于评测机性能原因,评测时题目时间限制可能会进行调整。\par
	3. 时间限制、空间限制分别不小于标准程序运行时间和内存使用的 1.5 倍、1 倍。\par
	4. 本场比赛仅支持 C++14 语言。\par
	5. \udot{不保证难度有序。} \par
	6. 本校模拟赛题不得外传,包括但不限于:私自拷题,在任何网站记录题目信息(包括洛谷非公开云剪贴板,非公开题目等),向在线 AI 询问题目,或任何可能造成其他人员获取题目的操作。\par
	%%%%%%%%%%%%%%%%%%%%%%%%%%%%%%%%%%%%%%%%%%%%%%%%%%%%%%%%%%%%%%%%
	\newprob{Tribios}{Tribios} \\ \indent
	「汝将碎作千片,凋零在他乡的土壤。」
	\para{题目描述} \\ \indent
	定义:给定一个长度为 \( n \) 的数字序列 \( a_1, a_2, \ldots, a_n \),其中每个元素都是正整数。\par
	一个子数字序列(不要求连续) \( S \) 为“平衡数字序列”,\par 当且仅当对于任意两个不同的元素 \( a_i \) 和 \( a_{i+1} \),满足以下条件:
	\[
	\left| a_i - a_{i+1} \right| \leq \min(a_i, a_{i+1})
	\]
	
	特别地,如果 $S$ 中只有一个元素,那么认为 $S$ 也是平衡的。
	
	现在,你需要动态地处理以下两种操作:
	
	1. \textbf{插入操作}:在数字序列末尾放入一个值为 \( x \) 的数字。
	
	2. \textbf{查询操作}:查询当前序列中最多能留下多少数字,它们组成“平衡数字序列”。
	\para{输入格式} \\ \indent\par
	从文件 \filename{Tribios.in} 中读入数据。
	\begin{itemize}
		\item 第一行输入一个正整数 $T$。
		\item 然后有 $T$ 行,每行有两种情况:
		\begin{itemize}
			\item \texttt{1 x} 表示插入操作,其中 $x$ 的含义如上所示
			\item \texttt{2} 表示查询操作
		\end{itemize}
	\end{itemize}
	\para{输出格式} \\ \indent
	输出到文件 \filename{Tribios.out} 中。\par
	对每个查询操作输出一行,意义如题。
	\sample{1}{输入}
	\begin{lstlisting}
		3 
		1 1 
		1 1
		2
	\end{lstlisting}
	
	\sample{1}{输出}
	\begin{lstlisting}
		2
	\end{lstlisting}
	
	\para{数据范围} \\ \indent\par
	对于 $20\%$ 的数据,有 $T\leq 10^2$,\par
	对于另外 $20\%$ 的数据,有 $T\leq 10^4$,\par
	对于 $100\%$ 的数据,有 $T\leq 10^6,x\leq 10^5$。 \par
	
	%%%%%%%%%%%%%%%%%%%%%%%%%%%%%%%%%%%%%%%%%%%%%%%%%%%%%%%%%%%%%%%%
	\newprob{SkeMma720}{Anaxagoras} \\ \indent
	「汝将超越至纯粹之终极,回归腐败枯黑。」
	\para{题目描述} \\ \indent
	给定长度为 \(N\) 的数组 \(A\),处理 \(M\) 次操作:
	\begin{enumerate}
		\item \textbf{更新操作}:给定区间 \([l, r]\) 和值 \(x\),将区间内每个元素异或 \(x\)
		\item \textbf{查询操作}:给定区间 \([l, r]\),求所有非空子序列异或和之和,模 \(10^9+7\)
	\end{enumerate}
	\para{输入格式} \\ \indent
	从文件 \filename{Anaxagoras.in} 中读入数据。
	\begin{itemize}
		\item 第一行输入两个整数 \(N\) 和 \(M\)。
		\item 第二行输入 \(N\) 个整数,表示数组初始值。
	\end{itemize}
	
	接下来 \(M\) 行,每行格式为:
	\begin{itemize}
		\item \texttt{1 l r x} 表示更新操作($1 \leq l \leq r \leq N$)
		\item \texttt{2 l r} 表示查询操作($1 \leq l \leq r \leq N$)
	\end{itemize}
	\para{输出格式} \\ \indent
	输出到文件 \filename{Anaxagoras.out} 中。\par
	对每个查询操作输出一行,表示答案模 \(10^9+7\) 的结果。
	
	\sample{1}{输入}
	\begin{lstlisting}
		3 3
		1 2 3
		2 1 3
		1 1 3 4
		2 1 3	
	\end{lstlisting}
	
	\sample{1}{输出}
	\begin{lstlisting}
		12
		28	
	\end{lstlisting}
	\sample{1}{解释} \\ \indent
	\begin{itemize}
		\item 	初始状态
		
		数组初始值为 [1, 2, 3],执行第一次查询操作 2 1 3:
		\begin{itemize}
			\item 单元素子序列:1, 2, 3
			异或和分别为 1, 2, 3
			总和:1 + 2 + 3 = 6
			
			\item 两元素子序列:
			
			[1,2] → 1 XOR 2 = 3
			
			[1,3] → 1 XOR 3 = 2
			
			[2,3] → 2 XOR 3 = 1
			总和:3 + 2 + 1 = 6
			
			\item 三元素子序列:[1,2,3] → 1 XOR 2 XOR 3 = 0
		\end{itemize}
		
		总结果:6 + 6 + 0 = 12
		
		\item 更新操作后
		
		执行更新操作 1 1 3 4,数组变为 [5, 6, 7]\\
		(因为 1 XOR 4 = 5,2 XOR 4 = 6,3 XOR 4 = 7),再执行第二次查询操作 2 1 3:
		\begin{itemize}
			\item 单元素子序列:5, 6, 7
			异或和分别为 5, 6, 7
			总和:5 + 6 + 7 = 18
			
			\item 两元素子序列:
			
			[5,6] → 5 XOR 6 = 3
			
			[5,7] → 5 XOR 7 = 2
			
			[6,7] → 6 XOR 7 = 1
			总和:3 + 2 + 1 = 6
			
			\item 三元素子序列:[5,6,7] → 5 XOR 6 XOR 7 = 4
		\end{itemize}
		
		
		总结果:18 + 6 + 0 = 24
	\end{itemize}
	\para{数据范围} \\ \indent
	共十个测试点,启用捆绑测试:
	\[
	\begin{array}{cccc}
		\hline
		\texttt{测试点编号} & N \leq & M \leq & \texttt{特殊性质} \\ \hline
		1-3 & 10^3 & 10^3 & $A$ \\ 
		4-6 & 10^5 & 10^5 & $B$ \\ 
		7-10 & 10^5 & 10^5 & \texttt{无限制} \\ \hline
	\end{array}
	\]
	\textbf{保证所有输入值满足} \(0 \leq A_i, x < 2^{30}\),且 \(1 \leq l \leq r \leq N\)
	\begin{itemize}
		\item \textbf{特殊性质说明}:
		\begin{itemize}
			\item $A$:仅包含查询操作
			\item $B$:所有查询操作的区间均为整个数组
		\end{itemize}
	\end{itemize}
	%%%%%%%%%%%%%%%%%%%%%%%%%%%%%%%%%%%%%%%%%%%%%%%%%%%%%%%%%%%%%%%%%%%%%%%
	\newprob{EpieiKeia216}{Castorice}\par% \\ \indent
	「花海尽头,生者的魂灵将温暖汝之指尖......」\par
	「相拥过后,便是永恒的离别。」
	\para{题目描述} \\ \indent
	EpieiKeia216将接过「死亡」的权柄。\par
	在取得火种之前,她想为开拓者讲述有关她的 $n$ 段有重复的回忆。\par 但是时间有限,只够她讲述其中的 $m$ 段。\par 对于被选出来的第 $i$ 号回忆,其能提供的价值为这 $m$ 段回忆中第 $i$ 号回忆的出现次数。\par
	求EpieiKeia216能讲述的回忆的最大价值。
	\para{输入格式} \\ \indent
	从文件 \filename{Castorice.in} 中读入数据。 \par
	第一行两个整数 $n,m$\par
	第二行 $n$ 个整数,第 $i$ 个整数表示第 $i$ 段回忆的编号
	\para{输出格式} \\ \indent
	输出到文件 \filename{Castorice.out} 中。 \par
	一行一个整数,表示能讲述回忆的最大价值。
	
	\sample{1}{输入}
	\begin{lstlisting}
		15 10
		1 3 2 1 2 3 1 2 1 1 1 1 1 1 2
	\end{lstlisting}
	
	\sample{1}{输出}
	\begin{lstlisting}
		82
	\end{lstlisting}
	
	\sample{1}{解释} \\ \indent
	选取 $9$ 个 $1$ 号回忆,一个 $2$ 号回忆或者选取 $9$ 个 $1$ 号回忆,$1$ 个 $3$ 号回忆
	
	\para{数据范围} \\ \indent
	对于 $20\%$ 的数据,$n,m\le 20$。 \par 
	对于 $100\%$ 的数据,$n,m\le 2\times10^5,a_i\leq 2\times 10^5$。 \par 
	%%%%%%%%%%%%%%%%%%%%%%%%%%%%%%%%%%%%%%%%%%%%%%%%%%%%%%%%%%%%%%%%%%%%%%%%%%%
	\newprob{Hyacinthia}{Hyacinthia} \\ \indent
	「在彩虹桥的尽头,天空之子将缝补晨昏。」
	\para{题目描述} \\ \indent
	晨昏之眼的地图可以描述成一颗以 $1$ 为根,$n$ 个节点的树,第 $i$ 个节点上有 $a_i$ 个泰坦眷属。每个泰坦眷属可以选择向任意子节点走,直到走到叶子节点为止。\par
	每个叶子节点的承载能力都有限,为了缓解晨昏之眼的压力,你要使泰坦眷属最多的叶子节点的眷属个数尽量少,输出这个数字。
	
	\para{输入格式} \\ \indent
	从文件 \filename{Hyacinthia.in} 中读入数据。 \par
	第一行一个正整数 $n$ \par
	接下来一行 $n-1$ 个正整数,第 $i$ 表示节点 $i+1$ 的父亲\par
	接下来一行 $n$ 个正整数 $a_1,a_2,\dots,a_n$ 
	\para{输出格式} \\ \indent
	输出到文件 \filename{Hyacinthia.out} 中。 \par
	一行一个整数表示人最多的叶子节点的最少人数
	
	\sample{1}{输入}
	\begin{lstlisting}
		3
		1 1 
		3 1 2
	\end{lstlisting}
	
	\sample{1}{输出}
	\begin{lstlisting}
		3
	\end{lstlisting}
	
	\sample{1}{解释} \\ \indent
	$2$ 号节点最后具有 $1$ 号节点的 $2$ 个人,$2$ 号节点的 $1$ 个人。\par
	$3$ 号节点最后具有 $1$ 号节点的 $1$ 个人,$3$ 号节点的 $2$ 个人。
	
	
	\para{数据范围} \\ \indent
	\begin{center}
		\begin{tabular}{ccccc}
			\hline
			subtask & $n,m\leq$ & $0\leq a_i\leq$ & 特殊性质 &子任务分值\\
			\hline
			$1$ & $5$ & $5$ &无 & $10$ \\%×≤
			$2$ & $200$ & $10^9$ &无& $10$ \\
			$3$ & $2000$ & $10^9$&无 & $30$\\
			$4$ & $2\times 10^5$ &$10^9$&图是链 & $10$\\
			$5$ & $2\times 10^5$ &$10^9$&图是菊花& $10$ \\
			$6$ & $2\times 10^5$ &$10^9$&无& $30$ \\
			\hline
		\end{tabular}
	\end{center}
	%%%%%%%%%%%%%%%%%%%%%%%%%%%%%%%%%%%%%%%%%%%%%%%%%%%%%%%%%%%%%%%%%%%%%%%%%%%
	\newprob{NeiKos496}{Phainon} \\ \indent
	「汝将肩负骄阳,直至灰白的黎明显著」
	\para{题目描述} \\ \indent
	NeiKos096 开始了永劫回归\par
	但是永劫回归对精神磨损严重,容易被「毁灭」冲毁理性\par
	开始时,NeiKos096 可以将理性值设置为 $[L,R]$ 中的任意正整数\par
	每次轮回末,理性值 $i$ 变为 $n$ 除以 $i$ 的余数\par
	当理性值为 $0$ 时,认为 NeiKos096 失去了理性\par
	你的任务是对于给定的 $n$,以及 $T$ 组 $L$,$R$,告诉 NeiKos096 最多能保持多少个轮回的理性。
	\para{输入格式} \\ \indent
	从文件 \filename{Phainon.in} 中读入数据。 \par
	第一行,两个正整数,代表 $n, T$。\par
	第二行至第 $T + 1$ 行,每行两个正整数,代表每一组 $L, R$。
	\para{输出格式} \\ \indent
	输出到文件 \filename{Phainon.out} 中。 \par
	
	共 $T$ 行,每行一个正整数,代表每次询问的结果。
	
	\sample{1}{输入}
	\begin{lstlisting}
		9 1 
		1 8
	\end{lstlisting}
	
	\sample{1}{输出}
	\begin{lstlisting}
		3
	\end{lstlisting}
	
	\sample{1}{解释} \\ \indent
	令永劫回归开始时 $i=5$ ,\par 则此时第一轮回末 $i=4$ ,\par 第二轮回末 $i=1$ ,\par 第三轮回末 $i=0$。\par 所以,NeiKos096 在第三轮回末之前都可以保持理性。
	
	
	\para{数据范围} \\ \indent
	本题启用捆绑测试
	\begin{center}
		\begin{tabular}{ccccc}
			\hline
			subtask & 数据点 & $n$ & $T$ & 特殊性质 \\
			\hline
			$1$ & $1$ & $\leq 2 \times 10^2$ &  $\leq3 \times 10^3$ & $A$ \\%×≤
			$2$ & $2-3$ & $\leq 2 \times 10^2$ &  $\leq3 \times 10^3 $& 无 \\
			$3$ & $4$ & $\leq 5 \times 10^5 $&$ \leq 5\times10^5$ & $A$ \\
			$4$ & $5-10$ &$\leq  5 \times 10^5 $& $\leq 5 \times 10^5$ & 无 \\
			\hline
		\end{tabular}
	\end{center}
	特殊性质 $A$: $L = 1$。
	%%%%%%%%%%%%%%%%%%%%%%%%%%%%%%%%%%%%%%%%%%%%%%%%%%%%%%%%%%%%%%%%%%%%%%%%%%%
	\newprob{PhiLia093}{Cyrene} \\ \indent
	「当然, 这一定是个不同以往的浪漫故事…… 你也是这么想的,对吧♪」
	\para{题目描述} \\ \indent
	PhiLia093 凭借岁月的权柄与 NeiKos496 一起进行永劫回归。\par
	具体而言,一条时间线上面有 $n$ 件事,事件种类为 $a_i$,其中 $1\leq a_i\leq n$\par
	第 $i$ 次永劫回归结束时刚经历完第 $p_i$ 件事,\par 
	每次永劫回归开始都会回到第 $1$ 件事,\par
	这样的永劫回归进行了 $m$ 次,\par
	最后一次,也就是第 $m+1$ 次永劫回归,经历了完整的 $n$ 件事。\par
	设种类为 $i$ 的事件经历了 $z_i$ 次,\par
	现在 PhiLia093 想知道 $(z_1\times 1) \oplus (z_2\times 2) \oplus \dots \oplus (z_n\times n)$
	
	\para{输入格式} \\ \indent
	从文件 \filename{Cyrene.in} 中读入数据。 \par
	第一行一个整数 $n$。\par
	第二行 $n$ 个整数 $a_i$。\par
	第三行一个整数 $m$。\par
	第四行 $m$ 个整数 $p_i$。 
	\para{输出格式} \\ \indent
	输出到文件 \filename{Cyrene.out} 中。 \par
	输出一行一个整数,含义如题目所示。
	
	\sample{1}{输入}
	\begin{lstlisting}
		5
		1 2 3 4 1
		3
		2 1 2
	\end{lstlisting}
	
	\sample{1}{输出}
	\begin{lstlisting}
		4
	\end{lstlisting}
	
	\sample{1}{解释} \\ \indent
	种类为 $1\sim 5$ 的事件出现次数为 $5,3,1,1,0$。
	
	\para{数据范围} \\ \indent
	对于 $100\%$ 的数据,$1\leq n\leq 10^6 ,0\le m\le 10^6,1\le a_i,p_i\le n$。  \par 
	对于 $40\%$ 的数据,$n,m\le 10^3$。 \par 
	对于 $60\%$ 的数据,$n,m\le 10^5$。 \par 
	对于另外 $20\%$ 的数据,$m=0$。 \par 
\end{document}